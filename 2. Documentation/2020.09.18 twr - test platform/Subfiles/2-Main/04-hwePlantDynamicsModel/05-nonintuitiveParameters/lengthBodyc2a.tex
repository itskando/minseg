% !TEX spellcheck = English (United States) (Aspell)
% !TEX TS-program = arara
%  arara: lmkclean
%  arara: pdflatex: {   draft: yes, options: '-file-line-error -halt-on-error' }
%  arara: biber
%  arara: pdflatex: {   draft: yes, options: '-file-line-error -halt-on-error' }
%  arara: pdflatex: { synctex: yes, options: '-file-line-error -halt-on-error' }
%  arara: lmkclean
\documentclass[crop=false,float=true,class=scrreprt]{standalone}

\providecommand{\main}{../../../..}
% Preamble
 % meta tools
\usepackage{standalone}                 % allows for independent runs of subfiles. [includes currfile].
                                        % IMPORTANT: [realmainfile] option of currfile requires compiler option -recorder to be active.
                                        % IMPORTANT: standalone loads currfile package without options.  To configure currfile options:
                                        %            place them in \documentclass[options](standalone) options. 
                                        %      NOTE: Loading the currfile package with options will clash with the standalone package.
\usepackage{etoolbox}                   % allows if/else statements in code. [required for: standalone+nocite fix, equation numbering.]

% math
\usepackage{mathtools}                  % includes amsmath, supplements it.
\usepackage{subdepth}                   % allows manual vertical alignment for misaligned subscripts.

% font
\newcommand{\xLanguage}{american}       % do not remove: used in multiple locations.
\usepackage[\xLanguage]{babel}          % hyphenates words correctly, based on document language. [\xlanguage is defined immediately above.]
\usepackage{amssymb}                    % symbols. [permits use of \bigstar].
%\usepackage{mathptmx}                  % times new roman font. totally lame, but required by EB. [supercedes 'times' package].
\usepackage[T1]{fontenc}                % improves pdf reader's ability to copy    atypical text characters from output pdf file.
\usepackage[latin1]{inputenc}           % improves tex editor's ability to compile atypical text characters into output pdf file from tex file.

%\usepackage{verbatim}                  % [superceded by listings] improves verbatim environment. [includes comment package].
\usepackage{listings}                   % improved version of verbatim. imports script languages (with syntax coloring).

\usepackage{color}                      % color commands.
\usepackage[dvipsnames]{xcolor}         % additional color commands.

\usepackage{soul}                       % strikethrough commands.
\usepackage{transparent}                % transparent objects/font.

% layout: page/spacing/headings
\usepackage{geometry}                   % margin/page layout settings.
\usepackage{changepage}                 % allows adjustwidth, for figures larger than the margins.
\usepackage{pdflscape}                  % landscape page layout.
\usepackage{scrlayer-scrpage}           % improved header commands. [supercedes `fancyhdr' package].

\usepackage{eso-pic}                    % watermarks. [xwatermark is not compatible with scrpage. draftwatermark is not included with Tex Live.]

 \usepackage{setspace}                  % line spacing (allows \doublespacing). <-- not for urithesis
%\usepackage{parskip}                   % [superceded by KOMA]. tidies spacing. 

%\usepackage{titlesec}                  % [superceded by KOMA]. change style of all page      headers, etc.
%\usepackage{sectsty}                   % [superceded by KOMA]. change style of all sectional headers, etc.
\usepackage[toc,page]{appendix}         % appendices. [toc adds Appendices to TOC, page adds a page listed Appendices in the document.]

% references
\usepackage{chngcntr}                   % allows changes mid-document to section depth of equation counter reset.
\usepackage[titles]{tocloft}            % allows new lists such as \listofequations and \listoflistings.
                                        % Note: titles option permits header on table of contents and list of figures/tables/etc pages.
\usepackage{titletoc}                   % allows sub-[tables of contents]. allows increased margin between numbers and labels in toc.
                                        % IMPORTANT: [breaks \section* commands. use \setcounter{secnumdepth}{0} instead.]

% floats: figures/tables/lists
\usepackage{float}                      % improves floating objects (graphics/tables).

\usepackage{graphics}
\usepackage{graphicx}                   % graphics incorporation.
\usepackage{wrapfig}                    % allows text wrapping of figures.
\usepackage{subcaption}                 % allow captions with the subcaption command. [automatically loads caption package.]

\usepackage{tabu}                       % improved table commands.
\usepackage{longtable}                  % allows multipage tables.
\usepackage{multirow}                   % multirow command.
\usepackage{bigstrut}                   % bigstrut command. adds slight space up [t], down [b], or both. try next to \hline.
\usepackage{booktabs}                   % improved hline spacing commands in tables.

\usepackage{enumitem}                   % improved alterations to indent in enumerate/itemize. allows enumerate counter nesting.

% references
\usepackage{csquotes}                   % required for biblatex when using babel.
\usepackage{xpatch}                     % required for ieee-style @online no-date fix.

\usepackage[ backend      = biber       % supercedes bibtex
           , sorting      = none        % none: entries appear in order of \cite use
           , refsegment   = chapter     % restart numbering at beginning of each chapter
           , defernumbers = true,       % required for added global bibliography
           , style        = ieee        % ieee style
%          , doi          = false       % false for ieee style?
%          , isbn         = false       % false for ieee style?
           , citestyle    = numeric,    % \cite{REF:A, REF:B} => [1,2] instead of the default => [1], [2]
           , url          = true   ]
            {biblatex}                  % bibliographies (supercedes bibtex, biber, ...)

% hyperref                              % IMPORTANT: Always load hyperref last. It tends to break other packages when loaded first.
\usepackage[ hidelinks
           , linktoc   = all     ]
           {hyperref}                   % hyperlinks.












 %Setup Up Paths for Figures
\graphicspath{ %
             {\main/Figures/}                                % Directories for master file
             {\main/Figures/01-Intro/TitlePage/} 
             %
             {\main/Figures/02-Main/}
             {\main/Figures/02-Main/Verification/}
             %
             %
             }



 % Page size, margin, and header/footer settings:

% Enable "showframe" when editing margins
%\usepackage{showframe}                                            % Uncomment this to display header/footer/margins outlines.

% General margin settings:
 \newlength{\xhmargin   } \setlength  {\xhmargin   }{+1.000in    }
 \newlength{\xlmargin   } \setlength  {\xlmargin   }{\xhmargin   }
%                         \addtolength{\xlmargin   }{+0.500in    } % Include this row for additional margin for paper binding.
 \newlength{\xrmargin   } \setlength  {\xrmargin   }{\xhmargin   }
  
 \newlength{\xtmargin   } \setlength  {\xtmargin   }{+1.000in    } % Actual  tmargin = [\xtmargin - \xheadheight] = [0.500in] 
 \newlength{\xbmargin   } \setlength  {\xbmargin   }{+1.000in    } % Actual  bmargin = [\xbmargin - \xfootskip  ] = [0.500in] 

% Header  margin settings:
 \newlength{\xheadheight} \setlength  {\xheadheight}{+0.500in    } % May need to adjust tmargin in addition to this.
 \newlength{\xheadsep   } \setlength  {\xheadsep   }{+1.000em    }

% Footer  margin settings:
 \newlength{\xfootheight} \setlength  {\xfootheight}{+0.500in    } % May need to adjust bmargin in addition to this.
 \newlength{\xfootskip  } \setlength  {\xfootskip  }{\xfootheight} % Actual footskip = [\xfootheight - 1em + desired footsep] = [0.500in]
                          \addtolength{\xfootskip  }{-1.000em    } % <-- do not edit
                          \addtolength{\xfootskip  }{+1.000em    } % <-- do     edit: desired footsep



\KOMAoptions{headheight    = \xheadheight,
             footheight    = \xfootheight,
             DIV           = current,
             fontsize      = 10pt,
             parskip       = half-,
             toc           = chapterentrydotfill,
           % headings      = small,
           % headings      = openany,
             headings      = twolinechapter}

\geometry{letterpaper,
          tmargin       = \xtmargin,
          bmargin       = \xbmargin,
          lmargin       = \xlmargin,
          rmargin       = \xrmargin,
          headsep       = \xheadsep,
          footskip      = \xfootskip}

\savegeometry{default}




% Initialize double-spacing
\doublespacing




% Section headings

%\setkomafont{sectioning}   {\bfseries}
%\setkomafont{chapter}      {\nms}
%\setkomafont{section}      {\nms}
%\setkomafont{subsection}   {\nms}
%\setkomafont{subsubsection}{\nms}
%\setkomafont{paragraph}    {\nms}
%\setkomafont{subparagraph} {\nms}

\RedeclareSectionCommands[font       =  \nms,
                          beforeskip =  0pt,            % vspace before  chapter heading
                          innerskip  = -\parskip,       % vspace between chapter heading lines
                          afterskip  =  1\baselineskip] % vspace after   chapter heading
                         {chapter}
                         
\RedeclareSectionCommands[font       =  \nms,
                          afterskip  =  0.001em] % 0em puts the text inline with the heading. >.<
                          {section,subsection,subsubsection,paragraph,subparagraph}

% Make the word Chapter uppercase (but not the section heading label)
% Add a visible horizontal line
\renewcommand*{\chapterformat}{%
  \MakeUppercase{\chapappifchapterprefix{\nobreakspace}}\thechapter\autodot%
    \IfUsePrefixLine{%
        \par\nobreak\vspace*{-\parskip}\vspace*{-.6\baselineskip}%
        \rule{0.75\textwidth}{.5pt}%
}{\enskip}}%


\renewcommand\raggedchapter{\centering}


% Initialize headers and footers

\setkomafont{pageheadfoot}{\normalfont\normalcolor}

% Header: Center:
\chead{\fns}

% Footer: Center:
\cfoot{\thepage}

% Footer: Right-Side:
\ofoot{\fns}

% Header/footer enable/disable switch
\newpairofpagestyles{trueempty}{}
%  Enable: \pagestyle{scrheadings} [this is the default.]
% Disable: \pagestyle{trueempty}




% Watermark
 % Initialize watermark

\iffalse % <-- [ \iffalse: disabled | \iftrue: enabled ]
\AddToShipoutPictureFG{ 

\raisebox{.5\paperheight}{
\begin{minipage}[c][\paperheight][c]{\paperwidth}
\centering
\rotatebox[origin=c]{45}{ 
\scalebox{15}{ 
\transparent{0.35} \color{gray} D\sc{raft}
} } 
\end{minipage}
}

}
\fi


































% Floats:
 % Names of [float] and List of [float]

% Allow multiline captions to be centered.
\captionsetup{format=plain, justification=centering}




%-List of Contents
\expandafter\addto\csname captions\xLanguage\endcsname{% This line is needed when using the babel package.
  \renewcommand{\contentsname}{Table of Contents}%      % \xLanguage is defined in [1-packages.tex] near babel package.
                                                      }%  Renames Contents to List of Contents      

%-List of Code Listings
\renewcommand{\lstlistingname}    {Code Listing}              % Rename ``Listing'' floats to ``Code Listing'' floats.
\renewcommand{\lstlistlistingname}{List of \lstlistingname s \vspace{+0.40em}}








% Settings for importing scripts [listings package]


\lstset{ %
   backgroundcolor  = \color{white},     % choose the background color; you must add \usepackage{color} or \usepackage{xcolor}
   basicstyle       = \footnotesize,     % the size of the fonts that are used for the code
   breakatwhitespace= false,             % sets if automatic breaks should only happen at whitespace
   breaklines       = true,              % sets automatic line breaking
   captionpos       = tb,                % sets the caption-position
   commentstyle     = \color{Green},     % comment style
   deletekeywords   = {...},             % if you want to delete keywords from the given language
   escapeinside     = {\%*}{*)},         % if you want to add LaTeX within your code
   extendedchars    = true,              % lets you use non-ASCII characters; for 8-bits encodings only, does not work with UTF-8
   frame            = single,            % adds a frame around the code
   keepspaces       = true,              % keeps spaces in text, 
                                         %   useful for keeping indentation of code (possibly needs columns=flexible)
   keywordstyle     = \color{blue},      % keyword style
   language         = Matlab,            % the language of the code
   morekeywords     = {*,...},           % if you want to add more keywords to the set
   numbers          = left,              % where to put the line-numbers; possible values are (none, left, right)
   numbersep        = 5pt,               % how far the line-numbers are from the code
   numberstyle      = \tiny\color{gray}, % the style that is used for the line-numbers
   rulecolor        = \color{black},     % if not set, the frame-color may be changed on line-breaks within 
                                         %   not-black text (e.g. comments (green here))
   showspaces       = false,             % show spaces everywhere adding particular underscores; it overrides 'showstringspaces'
   showstringspaces = false,             % underline spaces within strings only
   showtabs         = false,             % show tabs within strings adding particular underscores
   stepnumber       = 1,                 % the step between two line-numbers. If it's 1, each line will be numbered
   stringstyle      = \color{purple},    % string literal style
   tabsize          = 2,                 % sets default tabsize to 2 spaces
   title            = \lstname           % show the filename of files included with \lstinputlisting; also try caption 
                                         %   instead of title
}















%% Sectioned Float Counters:

% Required Packages:
% - chngcntr
% - etoolbox
% - listings
% - titletoc


% Save a copy of the original float counters.
\let\xtheequationOriginal\theequation
\let\xthefigureOriginal\thefigure
\let\xthetableOriginal\thetable
\let\xthelstlistingOriginal\thelstlisting




% Command: Sectioned Counter

\newcommand{\sectionedCounter}     [1]
  % Input #1: section, subsection, subsubsection, paragraph, subparagraph, or \determineSection
  %
  %-This command configures all float counters to reset when switching to 
  %   a new section at (or above) a user-specified depth.
  % Reuse of the command overwrites the previous reset flags.
  %
  %-This command prepends all float counters with the current section number 
  %  (at the user-specified depth).
{
  \counterwithout*{equation}  {section} 
  \counterwithout*{figure}    {section}
  \counterwithout*{table}     {section}
  \counterwithout*{lstlisting}{section}
    
  \counterwithout*{equation}  {subsection} 
  \counterwithout*{figure}    {subsection}
  \counterwithout*{table}     {subsection}
  \counterwithout*{lstlisting}{subsection}
    
  \counterwithout*{equation}  {subsubsection} 
  \counterwithout*{figure}    {subsubsection}
  \counterwithout*{table}     {subsubsection}
  \counterwithout*{lstlisting}{subsubsection}
    
  \counterwithout*{equation}  {paragraph} 
  \counterwithout*{figure}    {paragraph}
  \counterwithout*{table}     {paragraph}
  \counterwithout*{lstlisting}{paragraph}
    
  \counterwithout*{equation}  {subparagraph} 
  \counterwithout*{figure}    {subparagraph}
  \counterwithout*{table}     {subparagraph}
  \counterwithout*{lstlisting}{subparagraph}

  \counterwithin*{equation}  {#1}   % Reset counter whenever there is a new \section
  \counterwithin*{figure}    {#1}
  \counterwithin*{table}     {#1}
  \counterwithin*{lstlisting}{#1}   % The listings counter is not actually defined until \AtBeginDocument. 
                                    % Thus, if using this command (including listings) within the preamble, 
                                    %   use ``\AtBeginDocument{\sectionedCounter{<input>}}'' instead.

  \renewcommand{\theequation  }{\sectionedCounterStyle{#1}{equation}}
  \renewcommand{\thefigure    }{\sectionedCounterStyle{#1}{figure}}
  \renewcommand{\thetable     }{\sectionedCounterStyle{#1}{table}}
  \renewcommand{\thelstlisting}{\sectionedCounterStyle{#1}{lstlisting}}  
}


\newcommand{\sectionedCounterStyle}[2]
% Input #1: section, subsection, subsubsection, paragraph, subparagraph
% Input #1: equation, lstlisting, table, figure
%
%-This command sets the syntax of float counters.
%
%-This command prepends the float counter number with a section number (at a user defined depth).
%-This command separates the float number and the section number with an en–dash.
%
%-This command takes <float> rather than \the<float> such that the input of \sectionedCounter may be used.
%
% Example:  Using \sectionedCounterStyle{subsection}{figure},
%           Figure~\ref{FIG:exampleFig} with the seventh figure in Section 1.3.4.5 outputs: Figure 1.3–7 .
{\csname the#1\endcsname--\arabic{#2}}

%{\csname the#1\endcsname--\ifnum\value{#2}<10 0\fi\arabic{#2}} <-- Method to zero pad the float number.



% Provide extra \hspace in Lists of <Float>s for the increased number of characters in the counters.
\newlength{\xtocmargin    } \setlength{\xtocmargin    }{3.5em}
\newlength{\xtoclabelwidth} \setlength{\xtoclabelwidth}{3.5em}
\newlength{\xlsttocmargin } \setlength{\xlsttocmargin }{0.0em} % Needs to be \xtoclabelwidth - \xtocmargin.


%\dottedcontents{section}[margin from leftmargin]{above-code}{label width}{leader width}
 \dottedcontents{figure}[\xtocmargin]{}{\xtoclabelwidth}{1pc}     % No spaces allowed.
 \dottedcontents {table}[\xtocmargin]{}{\xtoclabelwidth}{1pc}     % No spaces allowed.

\makeatletter
\renewcommand*{\l@lstlisting}[2]{\@dottedtocline{1}{\xlsttocmargin}{\xtoclabelwidth}{#1}{#2}}
\makeatother




% Set float counters to include their full section number.
\AtBeginDocument{\sectionedCounter{subsection}}

% Recall:
% The listings counter is not actually defined until \AtBeginDocument. 
% Thus, if using this command (including listings) within the preamble, 
%   use ``\AtBeginDocument{\sectionedCounter{<input>}}'' instead.

% Command: Determine Section
\newcommand{\determineSection}{%  [Provides full section counter of current section, independent of the section depth.]
  \ifnum\value{subsubsection} > 0%
  \ifnum\value{paragraph}     > 0% 
  \ifnum\value{subparagraph}  > 0 paragraph%
  \else subsubsection\fi%
  \else subsection\fi%
  \else section\fi%
}

































% Commands:
 % Command: Multicol / Multirow
\newcommand{\mc}	[3]	{\multicolumn{#1}{#2}{#3}}                 % Abbreviates multicolumn. [n.col][ alignments ][content]
\newcommand{\mr}	[3]	{\multirow{#1}{#2}{#3}}                    % Abbreviates multirow   . [n.row][width(use *)][content]

% Command: Font Changes
\newcommand{\Hgs}          {\Huge}                                   % Abbreviates Huge     size     font command.
\newcommand{\hgs}          {\huge}                                   % Abbreviates huge     size     font command.
\newcommand{\LGs}          {\LARGE}                                  % Abbreviates LARGE    size     font command.
\newcommand{\Lgs}          {\Large}                                  % Abbreviates Large    size     font command.
\newcommand{\lgs}          {\large}                                  % Abbreviates large    size     font command.

\newcommand{\nms}          {\normalsize}                             % Abbreviates normal   size     font command.
\newcommand{\sms}          {\small}                                  % Abbreviates small    size     font command.
\newcommand{\fns}          {\footnotesize}                           % Abbreviates footnote size     font command.
\newcommand{\scs}          {\scriptsize}                             % Abbreviates script   size     font command.

\newcommand{\tnf}	[1]	{\textnormal{#1}}                         % Abbreviates text   normal     font command.
\newcommand{\tbf}	[1]	{\textbf{#1}}                             % Abbreviates text   bold       font command.
\newcommand{\tif}	[1]	{\textit{#1}}                             % Abbreviates text   italics    font command.
\newcommand{\tuf}	[1]	{\ul{#1}}                                 % Abbreviates text   underline  font command.
\newcommand{\ttt}	[1]	{\texttt{#1}}                             % Abbreviates text   teletype   font command. [monospace font.]

\newcommand{\sbf}	[1]	{\boldsymbol{#1}}                         % Abbreviates symbol bold       font command.
\newcommand{\mbf}	[1]	{\mathbf{#1}}                             % Abbreviates math   bold       font command.
\newcommand{\mrm}	[1]	{\mathrm{#1}}                             % Abbreviates math   roman      font command.


































 % Section numbering: Table of contents and section depth
\newcounter{xTocdepth}    \setcounter{xTocdepth}   {2} % These are used in multiple locations.
\newcounter{xSecnumdepth} \setcounter{xSecnumdepth}{5} % These are used in multiple location.

\setcounter{tocdepth}    {\thexTocdepth}
\setcounter{secnumdepth} {\thexSecnumdepth}




% Command: Subsection 
% -[Interchange paragraph and subparagraph with their subsubsubparagraph and subsubsubsub paragraph equivalents].
\newcommand{\subsubsubsection}     [1] {                                \paragraph{#1}                                            }
\newcommand{\subsubsubsectionA}    [1] { \setcounter{secnumdepth}{0}    \paragraph{#1} \setcounter{secnumdepth}{\thexSecnumdepth} }
\newcommand{\paragraphA}           [1] { \setcounter{secnumdepth}{0}    \paragraph{#1} \setcounter{secnumdepth}{\thexSecnumdepth} }
\newcommand{\subsubsubsubsection}  [1] {                             \subparagraph{#1}                                            }
\newcommand{\subsubsubsubsectionA} [1] { \setcounter{secnumdepth}{0} \subparagraph{#1} \setcounter{secnumdepth}{\thexSecnumdepth} }
\newcommand{\subparagraphA}        [1] { \setcounter{secnumdepth}{0} \subparagraph{#1} \setcounter{secnumdepth}{\thexSecnumdepth} }


% Format paragraph and subparagraph exactly like subsection.
% -subsubsection is formatted like subsection by default.
\makeatletter

\renewcommand{\paragraph}               %
  {\@startsection{paragraph}{4}{\z@}    %
  {-2.5ex\@plus -1ex \@minus -.25ex}    %
  {1.25ex \@plus .25ex}                 %
  {\normalfont\sffamily\normalsize\bfseries}     }

\renewcommand{\subparagraph}            %
  {\@startsection{subparagraph}{5}{\z@} %
  {-2.5ex\@plus -1ex \@minus -.25ex}    %
  {1.25ex \@plus .25ex}                 %
  {\normalfont\sffamily\normalsize\bfseries}     }

\makeatother













 % Command: Change page size
\newcommand{\beginLargePage}[2]{
  %\pdfpagewidth  = 11in       % [\pdfpagewidth and \pdfpageheight are superceded by KOMAoptions].
  %\pdfpageheight = 17in
  \KOMAoptions{paper        = #1:#2        , % Inputs are measurements:
               pagesize                    , % Example A: \beginLargePage{11in}{17in}
               headheight   = \xheadheight , % Example B: \beginLargePage{08in}{14in}
               footheight   = \xfootheight ,
               DIV          = current      }
               
  \newgeometry{layoutwidth  = #1         ,
               layoutheight = #2         , 
               tmargin      = \xtmargin  ,
               bmargin      = \xbmargin  ,
               hmargin      = \xhmargin  ,
               headsep      = \xheadsep  ,
               footskip     = \xfootskip }
                            }




% Command: Revert page size
\newcommand{\stopLargePage}{
  %\pdfpagewidth  = 08.5in      % [\pdfpagewidth and \pdfpageheight are superceded by KOMAoptions].
  %\pdfpageheight = 11.0in
  \KOMAoptions{paper=8.5in:11in,pagesize,DIV=current}
  \restoregeometry
                           }

% Bug Fixes:
 \iffalse

% Allow \nocite with standalone package
\makeatletter
\def\@documentnocite#1{\@bsphack
  \@for\@citeb:=#1\do{%
    \edef\@citeb{\expandafter\@firstofone\@citeb}%
    \if@filesw\immediate\write\@auxout{\string\citation{\@citeb}}\fi
    \@ifundefined{b@\@citeb}{\G@refundefinedtrue
      \@latex@warning{Citation `\@citeb' undefined}}{}}%
  \@esphack}
\AtBeginDocument{\let\nocite\@documentnocite}
\makeatother
% [Ideally \nocite will be patched in a later distribution.]

\fi






































  % Preamble [document configuration]
axle
\begin{document}

\subsection{Length From Body Center of Mass to Body Axis of Rotation $l_{b.c2a}$}
\label{SEC:hwePlantDynamicsModel:nonintuitiveParameters:lengthBodyc2a}

The length from the body center of mass to the body axis of rotation $l_{b.c2a}$
may be determined using more than one method.

\subsubsection{Yamamoto Method}
\label{SEC:hwePlantDynamicsModel:nonintuitiveParameters:lengthWheels2Body:yamamoto}

As seen in Figure~%
\ref{FIG:hwePlantDynamicsModel:isometric}
{\fns[\tif{on page}~\pageref{FIG:hwePlantDynamicsModel:isometric}]},
\textcite{REF:online:2009-yamamoto} 
assumes that the geometries of the wheels and the body are uniform.
He also assumes that the masses of these geometries are uniform.
He therefore defines 
length from the body center of mass to the body axis of rotation $l_{b.c2a}$,
as exhibited in Equation~%
\eqref{EQN:hwePlantDynamicsModel:nonintuitiveParameters:lengthWheels2Body:yamamoto}




\begin{equation}
\label{EQN:hwePlantDynamicsModel:nonintuitiveParameters:lengthWheels2Body:yamamoto}
\begin{array}{ccc}
l_{b.c2a}
& = &
\displaystyle\frac{l_{b.h}}{2}
\end{array}
\end{equation}




\clearpage




\subsubsection{Vaccaro Method}
\label{SEC:hwePlantDynamicsModel:nonintuitiveParameters:lengthWheels2Body:vaccaro}

Since the geometries of the actual hardware are assumed 
to significantly deviate from the assumption of uniform mass distribution,
an alternative method is instead used to calculate
length from the body center of mass to the body axis of rotation $l_{b.c2a}$,
as exhibited in Equation~%
\eqref{EQN:hwePlantDynamicsModel:nonintuitiveParameters:lengthWheels2Body:yamamoto}

If the hardware is mounted at both wheel axles 
\tif{along the axis which is shared by both wheel axles},
and if the hardware is given a degree of freedom to rotate about the wheel axle axis,
\tif{without rotating the actual wheel axles},
then the hardware may be lifted slightly and then released 
to swing freely {\fns like a pendulum} along that axis.

Allowing the hardware to freely swing like a pendulum along the wheel axle axis
significantly simplifies the dynamic equations of motion of the hardware.
Furthermore, if friction at the newly added mount coupling points is negligible,
then there will not be a need to model and implement the friction into the dynamics equations.

If the hardware is freely swung like a pendulum along the wheel axle axis as described above,
then the relations exhibited in Equations
\eqref{EQN:hwePlantDynamicsModel:nonintuitiveParameters:lengthWheels2Body:vaccaro:pendulumDynamics1}~%
-~%
\eqref{EQN:hwePlantDynamicsModel:nonintuitiveParameters:lengthWheels2Body:vaccaro:pendulumDynamics2}
become true.


\vspace{-3em}

\begin{align}
\label{EQN:hwePlantDynamicsModel:nonintuitiveParameters:lengthWheels2Body:vaccaro:pendulumDynamics1}
\theta  &= \phi_{x} \\[+0em]
\label{EQN:hwePlantDynamicsModel:nonintuitiveParameters:lengthWheels2Body:vaccaro:pendulumDynamics2}
\mbf{u} &= \mbf{0}
\end{align}

\vspace{-1em}

The effects of these changes are exhibited in Equation~%
\eqref{EQN:hwePlantDynamicsModel:nonintuitiveParameters:lengthWheels2Body:vaccaro:reducedDynamics1}
{\fns[\tif{on page}~%
\pageref{EQN:hwePlantDynamicsModel:nonintuitiveParameters:lengthWheels2Body:vaccaro:reducedDynamics1}%
]}.
This results in two relations, which are exhibited in Equation~%
\eqref{EQN:hwePlantDynamicsModel:nonintuitiveParameters:lengthWheels2Body:vaccaro:reducedDynamics2}.

\vspace{-0em}

\begin{equation}
\label{EQN:hwePlantDynamicsModel:nonintuitiveParameters:lengthWheels2Body:vaccaro:reducedDynamics2}
\begin{array}{ccccccccc}
\ddot{\phi}_{x}
& + &
&&
0
& = &
0
%%
\\[+1em]
%%
\ddot{\phi}_{x}
& + &
\displaystyle\frac{k_{1.5}}{k_{1.2} + k_{1.3}}
& \cdot &
\phi_{x}
& = &
0
&&
\Leftarrow
%%
\\[-1em]
%%
&&
\underbrace{
\hphantom{
\displaystyle\frac{k_{1.5}}{k_{1.2} + k_{1.3}}
}
}_{k_{\omega}}
\end{array}
\end{equation}

\vspace{-0em}

Of the two resulting relations in Equation~%
\eqref{EQN:hwePlantDynamicsModel:nonintuitiveParameters:lengthWheels2Body:vaccaro:reducedDynamics2},
the former cannot be true while the hardware is in motion;
thus, the latter is selected, as depicted on the right with a left-facing arrow.




\clearpage



\begin{landscape}

\vspace*{\fill}

\begin{equation}
\label{EQN:hwePlantDynamicsModel:nonintuitiveParameters:lengthWheels2Body:vaccaro:reducedDynamics1}
\begin{array}{ccccccccccc}
\mbf{K}_{1.\ddot{x}}
& \cdot 
\begin{bmatrix}
\ddot{\theta}  \\
\ddot{\phi}_{x}\\
\end{bmatrix}
& + &
\mbf{K}_{1.\dot{x}}
& \cdot 
\begin{bmatrix}
\dot{\theta}  \\
\dot{\phi}_{x}\\
\end{bmatrix}
& + &
\mbf{K}_{1.x}
& \cdot 
\begin{bmatrix}
\theta  \\
\phi_{x}\\
\end{bmatrix}
& = &
\mbf{K}_{1.v}
& \cdot
\begin{bmatrix}
v_{mtr.l}\\
v_{mtr.r}\\
\end{bmatrix}
%%
\\[-0em]
%%
&
\hphantom{\cdot}
\underbrace{
\hphantom{
\begin{bmatrix}
\ddot{\phi}_{x}\\
\end{bmatrix}
}
}_{}
&&&
\hphantom{\cdot}
\underbrace{
\hphantom{
\begin{bmatrix}
\ddot{\phi}_{x}\\
\end{bmatrix}
}
}_{}
&&&
\hphantom{\cdot}
\underbrace{
\hphantom{
\begin{bmatrix}
\ddot{\phi}_{x}\\
\end{bmatrix}
}
}_{}
&&&
\hphantom{\cdot}
\underbrace{
\hphantom{
\begin{bmatrix}
\ddot{\phi}_{x}\\
\end{bmatrix}
}
}_{}
%%
\\[+2em]
%%
\mbf{K}_{1.\ddot{x}}
& \cdot 
\begin{bmatrix}
\ddot{\phi}_{x}\\
\ddot{\phi}_{x}\\
\end{bmatrix}
& + &
\mbf{K}_{1.\dot{x}}
& \cdot 
\begin{bmatrix}
\dot{\phi}_{x}\\
\dot{\phi}_{x}\\
\end{bmatrix}
& + &
\mbf{K}_{1.x}
& \cdot 
\begin{bmatrix}
\phi_{x}\\
\phi_{x}\\
\end{bmatrix}
& = &
\mbf{K}_{1.v}
& \cdot
\begin{bmatrix}
\hphantom{v_{mtr.r}}\\[-2em]
0\\
0\\
\end{bmatrix}
%%
\\[+0.5em]
%%

\hphantom{\cdot}
\underbrace{
\hphantom{
\begin{bmatrix}
\ddot{\phi}_{x}\\
\end{bmatrix}
}
}_{}
&&&
\hphantom{\cdot}
\underbrace{
\hphantom{
\begin{bmatrix}
\ddot{\phi}_{x}\\
\end{bmatrix}
}
}_{}
&&&
\hphantom{\cdot}
\underbrace{
\hphantom{
\begin{bmatrix}
\ddot{\phi}_{x}\\
\end{bmatrix}
}
}_{}
&&&
\mc{2}{c}{
\underbrace{
\hphantom{
\begin{array}{cc}
\mbf{K}_{1.v}
& \cdot
\begin{bmatrix}
\hphantom{v_{mtr.r}}\\[-2em]
0\\
\end{bmatrix}
\end{array}
}
}_{\oslash}
}
%%
\\[+2.5em]
%%
\begin{bmatrix}
k_{1.1} & k_{1.2} \\
k_{1.2} & k_{1.3} \\
\end{bmatrix}
& \cdot 
\begin{bmatrix}
\ddot{\phi}_{x}\\
\ddot{\phi}_{x}\\
\end{bmatrix}
& + &
2 \cdot k_{1.4} \cdot
\begin{bmatrix}
+1 & -1 \\
-1 & +1 \\
\end{bmatrix}
& \cdot 
\begin{bmatrix}
\dot{\phi}_{x}\\
\dot{\phi}_{x}\\
\end{bmatrix}
& + &
\begin{bmatrix}
0 & 0 \\
0 & k_{1.5} \\
\end{bmatrix}
& \cdot 
\begin{bmatrix}
\phi_{x}\\
\phi_{x}\\
\end{bmatrix}
& = &
&
\hphantom{\cdot}
\begin{bmatrix}
\hphantom{v_{mtr.r}}\\[-2em]
0\\
0\\
\end{bmatrix}
%%
\\[+2em]
%%
\mc{2}{c}{
\underbrace{
\hphantom{
\begin{array}{cc}
\begin{bmatrix}
k_{1.1} & k_{1.2} \\
k_{1.2} & k_{1.3} \\
\end{bmatrix}
& \cdot 
\begin{bmatrix}
\ddot{\phi}_{x}\\
\ddot{\phi}_{x}\\
\end{bmatrix}
\end{array}
}
}_{}
}
&&
\mc{2}{c}{
\underbrace{
\hphantom{
\begin{array}{cc}
2 \cdot k_{1.4} \cdot
\begin{bmatrix}
+1 & -1 \\
-1 & +1 \\
\end{bmatrix}
& \cdot 
\begin{bmatrix}
\dot{\phi}_{x}\\
\dot{\phi}_{x}\\
\end{bmatrix}
\end{array}
}
}_{\oslash}
}
&&
\mc{2}{c}{
\underbrace{
\hphantom{
\begin{array}{cc}
\begin{bmatrix}
0 & 0 \\
0 & k_{1.5} \\
\end{bmatrix}
& \cdot 
\begin{bmatrix}
\phi_{x}\\
\phi_{x}\\
\end{bmatrix}
\end{array}
}
}_{}
}
%%
\\[+2em]
%%
\mc{2}{c}{
\begin{bmatrix}
\begin{pmatrix} k_{1.1} + k_{1.2} \end{pmatrix} \cdot \ddot{\phi}_{x}\\
\begin{pmatrix} k_{1.2} + k_{1.3} \end{pmatrix} \cdot \ddot{\phi}_{x}\\
\end{bmatrix}
}
& + &
&
\hphantom{\cdot}
\begin{bmatrix}
\hphantom{\dot{\phi_{x}}}\\[-2em]
0\\
0\\
\end{bmatrix}
& + &
\mc{2}{c}{
\begin{bmatrix}
0                      \\
k_{1.5} \cdot \phi_{x} \\
\end{bmatrix}
}
& = &
&
\hphantom{\cdot}
\begin{bmatrix}
\hphantom{v_{mtr.r}}\\[-2em]
0\\
0\\
\end{bmatrix}
%%
\end{array}
\end{equation}

\vspace*{\fill}

\end{landscape}




\clearpage




The coefficient term, abbreviated as $k_{w}$, is expanded in Equation~%
\eqref{EQN:hwePlantDynamicsModel:nonintuitiveParameters:lengthWheels2Body:vaccaro:reducedDynamics2:kw1}.
It may be expanded further with the use of Equation~%
\eqref{EQN:hwePlantDynamicsModel:nonintuitiveParameters:JPhiX},
as exhibited in Equation~%
\eqref{EQN:hwePlantDynamicsModel:nonintuitiveParameters:lengthWheels2Body:vaccaro:reducedDynamics2:kw2}.


\vspace{-1em}


\begin{gather}
\label{EQN:hwePlantDynamicsModel:nonintuitiveParameters:lengthWheels2Body:vaccaro:reducedDynamics2:kw1}
\begin{array}{ccccccccc}
k_{\omega}
& = &
\displaystyle\frac{k_{1.5}}{k_{1.2} + k_{1.3}}
& = &
\displaystyle\frac
{-m_{b} \cdot a_{g} \cdot l_{b.c2a}}
{
\begin{pmatrix} m_{b} \cdot r_{w} \cdot l_{b.c2a} \end{pmatrix}
+
\begin{pmatrix} m_{b} \cdot l_{b.c2a}^{2} + J_{b.\phi_{x}} \end{pmatrix}
}
\end{array}
%%
\\[+3em]
%%
\label{EQN:hwePlantDynamicsModel:nonintuitiveParameters:lengthWheels2Body:vaccaro:reducedDynamics2:kw2}
\begin{array}{ccccccccc}
k_{\omega}
& = &
\displaystyle\frac
{-m_{b} \cdot l_{b.c2a} \cdot a_{g}}
{
\begin{array}{cccccc}
m_{b} \cdot l_{b.c2a} \cdot r_{w}
& + &
m_{b} \cdot l_{b.c2a}^{2}
& + &
\begin{pmatrix} \displaystyle m_{b} \cdot l_{b.c2a}^{2} \cdot \frac{1}{3} \end{pmatrix}
\end{array}
}
& = &
\displaystyle\frac
{-a_{g} }
{
r_{w} + l_{b.c2a} \cdot 
\begin{pmatrix} \displaystyle 1 + \frac{1}{3} \end{pmatrix}
}
\end{array}
\end{gather}




\subsubsection*{Harmonic Oscillator}

Notably, the selected relation in Equation~%
\eqref{EQN:hwePlantDynamicsModel:nonintuitiveParameters:lengthWheels2Body:vaccaro:reducedDynamics2}
form-matches the equation for a harmonic oscillator
\cite[{\fns p. 119~-~120, 122~-~123}]{REF:textbook:1995-vaccaro},
as is exhibited in Equation~%
\eqref{EQN:hwePlantDynamicsModel:nonintuitiveParameters:lengthWheels2Body:vaccaro:harmonicOscillator}.


\vspace{-1em}

\begin{equation}
\label{EQN:hwePlantDynamicsModel:nonintuitiveParameters:lengthWheels2Body:vaccaro:harmonicOscillator}
\begin{array}{ccccccccc}
\ddot{y}
& + &
\omega^{2} 
& \cdot &
y
& = &
\omega^{2} 
& \cdot &
u
%%
\\[+1em]
%%
\ddot{\phi}_{x}
& + &
k_{\omega}
& \cdot &
\phi_{x}
& = &
k_{\omega}
& \cdot &
0
\end{array}
\end{equation}




This allows for the relation of the abbreviated term representing the system dynamics, $k_{w}$, 
to the natural angular frequency of the hardware {\fns[\tif{a pendulum}]} $\omega_{p}$, 
as is exhibited in Equation~%
\eqref{EQN:hwePlantDynamicsModel:nonintuitiveParameters:lengthWheels2Body:vaccaro:pendulumFrequency1}.


\begin{equation}
\label{EQN:hwePlantDynamicsModel:nonintuitiveParameters:lengthWheels2Body:vaccaro:pendulumFrequency1}
\begin{array}{ccccccccc}
\omega_{p}^{2}
& = &
k_{\omega}
& = &
\displaystyle\frac
{-a_{g} }
{
r_{w} + l_{b.c2a} \cdot \displaystyle\frac{4}{3}
}
\end{array}
\end{equation}




This proves significant since $\omega_{p}$ represents 
the angular frequency of the pendulum,
which is a measurable value,
and since $k_{\omega}$ includes the desired unknown term $l_{b.c2a}$.
{\fns[\tif{All other terms are known}]}.
The relation may rewritten to solve for
length from the body center of mass to the body axis of rotation $l_{b.c2a}$,
as is exhibited as Equation~%
\eqref{EQN:hwePlantDynamicsModel:nonintuitiveParameters:lengthWheels2Body:vaccaro:pendulumFrequency2}.





\begin{equation}
\label{EQN:hwePlantDynamicsModel:nonintuitiveParameters:lengthWheels2Body:vaccaro:pendulumFrequency2}
\begin{array}{ccccccccc}
l_{b.c2a}
& = &
\displaystyle -\frac{3}{4} 
\cdot
\begin{pmatrix}
\begin{array}{ccc}
\displaystyle \frac{a_{g}}{\omega_{p}^{2}} & + & r_{w}
\end{array}
\end{pmatrix}
& = &
\displaystyle -\frac{3}{4} 
\cdot
\begin{pmatrix}
\begin{array}{ccc}
\displaystyle \frac{a_{g}}
{\begin{pmatrix}2 \cdot \pi \cdot f_{p} \end{pmatrix}^{2}} & + & r_{w}
\end{array}
\end{pmatrix}
\end{array}
\end{equation}




\clearpage




\end{document}










