% Section numbering: Table of contents and section depth
\newcounter{xTocdepth}    \setcounter{xTocdepth}   {2} % These are used in multiple locations.
\newcounter{xSecnumdepth} \setcounter{xSecnumdepth}{5} % These are used in multiple location.

\setcounter{tocdepth}    {\thexTocdepth}
\setcounter{secnumdepth} {\thexSecnumdepth}




% Command: Subsection 
% -[Interchange paragraph and subparagraph with their subsubsubparagraph and subsubsubsub paragraph equivalents].
\newcommand{\subsubsubsection}     [1] {                                \paragraph{#1}                                            }
\newcommand{\subsubsubsectionA}    [1] { \setcounter{secnumdepth}{0}    \paragraph{#1} \setcounter{secnumdepth}{\thexSecnumdepth} }
\newcommand{\paragraphA}           [1] { \setcounter{secnumdepth}{0}    \paragraph{#1} \setcounter{secnumdepth}{\thexSecnumdepth} }
\newcommand{\subsubsubsubsection}  [1] {                             \subparagraph{#1}                                            }
\newcommand{\subsubsubsubsectionA} [1] { \setcounter{secnumdepth}{0} \subparagraph{#1} \setcounter{secnumdepth}{\thexSecnumdepth} }
\newcommand{\subparagraphA}        [1] { \setcounter{secnumdepth}{0} \subparagraph{#1} \setcounter{secnumdepth}{\thexSecnumdepth} }


% Format paragraph and subparagraph exactly like subsection.
% -subsubsection is formatted like subsection by default.
\makeatletter

\renewcommand{\paragraph}               %
  {\@startsection{paragraph}{4}{\z@}    %
  {-2.5ex\@plus -1ex \@minus -.25ex}    %
  {1.25ex \@plus .25ex}                 %
  {\normalfont\sffamily\normalsize\bfseries}     }

\renewcommand{\subparagraph}            %
  {\@startsection{subparagraph}{5}{\z@} %
  {-2.5ex\@plus -1ex \@minus -.25ex}    %
  {1.25ex \@plus .25ex}                 %
  {\normalfont\sffamily\normalsize\bfseries}     }

\makeatother












