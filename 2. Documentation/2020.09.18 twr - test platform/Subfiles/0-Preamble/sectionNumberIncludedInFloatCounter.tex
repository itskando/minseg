% Sectioned Float Counters:

% Required Packages:
% - chngcntr
% - etoolbox
% - listings
% - titletoc


% Save a copy of the original float counters.
\let\xtheequationOriginal\theequation
\let\xthefigureOriginal\thefigure
\let\xthetableOriginal\thetable
\let\xthelstlistingOriginal\thelstlisting




% Command: Sectioned Counter

\newcommand{\sectionedCounter}     [1]
  % Input #1: section, subsection, subsubsection, paragraph, subparagraph, or \determineSection
  %
  %-This command configures all float counters to reset when switching to 
  %   a new section at (or above) a user-specified depth.
  % Reuse of the command overwrites the previous reset flags.
  %
  %-This command prepends all float counters with the current section number 
  %  (at the user-specified depth).
{
  \counterwithout*{equation}  {section} 
  \counterwithout*{figure}    {section}
  \counterwithout*{table}     {section}
  \counterwithout*{lstlisting}{section}
    
  \counterwithout*{equation}  {subsection} 
  \counterwithout*{figure}    {subsection}
  \counterwithout*{table}     {subsection}
  \counterwithout*{lstlisting}{subsection}
    
  \counterwithout*{equation}  {subsubsection} 
  \counterwithout*{figure}    {subsubsection}
  \counterwithout*{table}     {subsubsection}
  \counterwithout*{lstlisting}{subsubsection}
    
  \counterwithout*{equation}  {paragraph} 
  \counterwithout*{figure}    {paragraph}
  \counterwithout*{table}     {paragraph}
  \counterwithout*{lstlisting}{paragraph}
    
  \counterwithout*{equation}  {subparagraph} 
  \counterwithout*{figure}    {subparagraph}
  \counterwithout*{table}     {subparagraph}
  \counterwithout*{lstlisting}{subparagraph}

  \counterwithin*{equation}  {#1}   % Reset counter whenever there is a new \section
  \counterwithin*{figure}    {#1}
  \counterwithin*{table}     {#1}
  \counterwithin*{lstlisting}{#1}   % The listings counter is not actually defined until \AtBeginDocument. 
                                    % Thus, if using this command (including listings) within the preamble, 
                                    %   use ``\AtBeginDocument{\sectionedCounter{<input>}}'' instead.

  \renewcommand{\theequation  }{\sectionedCounterStyle{#1}{equation}}
  \renewcommand{\thefigure    }{\sectionedCounterStyle{#1}{figure}}
  \renewcommand{\thetable     }{\sectionedCounterStyle{#1}{table}}
  \renewcommand{\thelstlisting}{\sectionedCounterStyle{#1}{lstlisting}}  
}


\newcommand{\sectionedCounterStyle}[2]
% Input #1: section, subsection, subsubsection, paragraph, subparagraph
% Input #1: equation, lstlisting, table, figure
%
%-This command sets the syntax of float counters.
%
%-This command prepends the float counter number with a section number (at a user defined depth).
%-This command separates the float number and the section number with an en–dash.
%
%-This command takes <float> rather than \the<float> such that the input of \sectionedCounter may be used.
%
% Example:  Using \sectionedCounterStyle{subsection}{figure},
%           Figure~\ref{FIG:exampleFig} with the seventh figure in Section 1.3.4.5 outputs: Figure 1.3–7 .
{\csname the#1\endcsname--\arabic{#2}}

%{\csname the#1\endcsname--\ifnum\value{#2}<10 0\fi\arabic{#2}} <-- Method to zero pad the float number.



% Provide extra \hspace in Lists of <Float>s for the increased number of characters in the counters.
\newlength{\xtocmargin    } \setlength{\xtocmargin    }{3.5em}
\newlength{\xtoclabelwidth} \setlength{\xtoclabelwidth}{3.5em}
\newlength{\xlsttocmargin } \setlength{\xlsttocmargin }{0.0em} % Needs to be \xtoclabelwidth - \xtocmargin.


%\dottedcontents{section}[margin from leftmargin]{above-code}{label width}{leader width}
 \dottedcontents{figure}[\xtocmargin]{}{\xtoclabelwidth}{1pc}     % No spaces allowed.
 \dottedcontents {table}[\xtocmargin]{}{\xtoclabelwidth}{1pc}     % No spaces allowed.

\makeatletter
\renewcommand*{\l@lstlisting}[2]{\@dottedtocline{1}{\xlsttocmargin}{\xtoclabelwidth}{#1}{#2}}
\makeatother




% Set float counters to include their full section number.
\AtBeginDocument{\sectionedCounter{subsection}}

% Recall:
% The listings counter is not actually defined until \AtBeginDocument. 
% Thus, if using this command (including listings) within the preamble, 
%   use ``\AtBeginDocument{\sectionedCounter{<input>}}'' instead.

% Command: Determine Section
\newcommand{\determineSection}{%  [Provides full section counter of current section, independent of the section depth.]
  \ifnum\value{subsubsection} > 0%
  \ifnum\value{paragraph}     > 0% 
  \ifnum\value{subparagraph}  > 0 paragraph%
  \else subsubsection\fi%
  \else subsection\fi%
  \else section\fi%
}































